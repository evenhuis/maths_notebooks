\documentclass{beamer}
\author{Alex Hiller}
\title{Python Training, MM1, MMS}
\setlength{\parindent}{0cm}
\setlength{\parskip}{0.25cm}

% Packages
\usepackage{amsmath}              % Mathematics
\usepackage{listings}             % Listings
\usepackage{color}                % Listings
\usepackage{courier}              % Listings
\usepackage{graphicx}
\usepackage{hyperref}

\begin{document}

\begin{frame}
  \maketitle
\end{frame}

\begin{frame}{Resources}
    All files are available at \url{github.com/xandhiller/maths_notebooks}. \par \pause
    I am contactable at \texttt{alexander.hiller@uts.edu.au}.\footnote{Usually respond in about 24 hours.} \par \pause
    Solutions can be provided if you get stuck. 
\end{frame}

%%%%%%%%%%%%%%%%%%%%
\begin{frame}{Relevant Files}
\texttt{%
00\_Python\_Basics.ipynb  \\ \pause
01\_Differentiation\_Integration.ipynb \\ \pause
02\_Vectors.ipynb \\ \pause
03\_Plotting.ipynb \\ \pause
04\_Matrices.ipynb }
\end{frame}
%%%%%%%%%%%%%%%%%%%%
\begin{frame}{Running the Files}
    \textless Microsoft Azure Demo \textgreater
\end{frame}
%%%%%%%%%%%%%%%%%%%%
\begin{frame}{Structure}
    \texttt{00\_Python\_Basics.ipynb} can be set as pre-requisite reading with the use of the setup document and video on how to run Azure Notebooks. \pause \par
    The remaining three for MM1 and MMS can be run in whatever order the coordinator sees fit. \pause
    \texttt{%
    01\_Differentiation\_Integration.ipynb \\ 
    02\_Vectors.ipynb \\ 
    03\_Plotting.ipynb \\ 
    04\_Matrices.ipynb }
\end{frame}
%%%%%%%%%%%%%%%%%%%%
\begin{frame}
    There also other notebooks that were developed for IMAM (37132). \par \pause
    These can be used by other courses, should you see fit. \par \pause
    \texttt{%
    01\_Differentiation\_Integration.ipynb \\ 
    02\_Vectors.ipynb \\ 
    03\_Plotting.ipynb \\ 
    04\_Matrices.ipynb \\
    05\_Riemann\_Sums.ipynb\\
    06\_Partial\_Derivatives.ipynb\\
    07\_Optimisation.ipynb\\
    08\_Vector\_Calculus.ipynb\\
    09\_Integrals\_and\_Animation.ipynb\\
    10\_Differential\_Equations.ipynb\\}
\end{frame}
%%%%%%%%%%%%%%%%%%%%
\begin{frame}{Demos}
    \textless Run through \texttt{Differentiation and Integration} \textgreater \par \pause
    \textless Run through \texttt{Matrices} \textgreater
\end{frame}
%%%%%%%%%%%%%%%%%%%%
\begin{frame}{Problem Solving}
    Produce a function that prints out odd values up to a number $n$. \par \pause
    Produce an implementation of Newton's Method.
\end{frame}
\end{document}
